\section{Procedimiento}

% \subsection{Análisis descriptivo de las variables}

% \begin{table}[!htbp]
% \centering
% \resizebox{\textwidth}{!}
%   {\begin{tabular}{ccccccccc}
%   \hline
%   Variable & Mínimo & Primer Cuartil & Mediana & Media   & Tercer Cuartil & Máximo & Desviación standar & Rango intercuartil \\
%   \hline
%   season   &        &        &         &        &         &        &  1.1148 &            \\
%   size     &        &        &         &        &         &        &  0.7371 &            \\
%   speed    &        &        &         &        &         &        &  0.9143 &            \\
%   mxPH     &  7.000 &  7.778 &  8.100  &  8.078 &   8.400 & 9.500  &  0.4717 &  0.6225    \\
%   mnO2     &  1.500 &  7.675 &  9.750  &  9.019 &  10.700 & 13.400 &  2.4072 &  3.025     \\
%   Cl       &  0.80  & 11.85  &  35.08  &  44.88 &  58.52  & 391.50 & 47.0670 & 46.6602    \\
%   NO3      &  0.050 &  1.364 &  2.820  &  3.384 &  4.540  & 45.650 &  3.8747 &   3.1757   \\
%   NH4      &  5.80  &  49.38 &  115.70 & 537.70 &  235.20 & 24060  & 2031.5848 & 185.875  \\
%   oPO4     &  1.25  & 18.56  & 46.28   & 78.27  & 102.80  & 564.60 & 92.6215 & 84.2657    \\
%   PO4      &  2.50  &  50.34 & 115.60  & 146.60 &  220.30 & 771.60 & 129.1081  & 169.91   \\
%   Chla     &  0.200 &  2.075 &  5.522  & 13.880 & 18.310 & 110.500 & 20.2646 & 16.2325    \\
%   n.a1.    & -33.2900 & -0.7231 &  11.5400 & 12.5600 & 25.0800 & 84.7600 & 22.3660 & 25.8000 \\
%   n.a2.    & -23.2200 & -0.5889 &   7.1610 &  6.5850 & 13.7900 & 34.5300 & 10.9065 & 14.3779 \\
%   n.a3.    & -15.7200 & -0.6968 &   4.9010 &  4.2800 &  8.5740 & 22.8000 &  7.1602 & 9.2707  \\
%   n.a4.    & -29.920  & -1.580  &   4.460  &  5.938  & 12.060  & 47.250  & 12.4270 & 13.6438 \\
%   \end{tabular}}
% \caption{Análisis descriptivo de todas las variables}
% \end{table}

% Hay distintos aspectos desctacables al observar el \emph{Cuadro 1} que
% representa el \textbf{Análisis descriptivo de todas las variables} en referencia a los
% estadisticos estudiados (Promedio, mediana, primer y tercer cuartil, mínimo,
% máximo, desviación estándar, y rango intercuartil):

% \begin{itemize}
%   \item Teniendo en cuenta las magnitudes, el máximo y el mínimo de cada una,
%   se puede considerar que para las variables su media y mediana están
%   relativamente cerca lo cual indica que los datos están esparcidos segun el
%   promedio de la muestra.
%   \item Para variables como \emph{Cl, NO3, NH4, Chla} se puede considerar que
%   la media constituye valores bajos por lo cual se puede evidenciar valores
%   singulares altos.
%   \item Las algas mas perjudiciales (n.a2. y n.a4.) tienen la segunda y cuarta
%   media mas alta entre los tipos de algas.
%   \item Cuando se observa el valor de los máximos de las algas mas perjudiciales
%    se oberva que n.a2. y n.a4. tienen cuarto y tercer máximo mas alto
%   \item La desviación estandar de NH4 es muy alta, lo cual nos puede llevar a
%   esperar valores dispersos respecto a su media en el todo el rango de los posibles
%   valores de la variable
% \end{itemize}

% \subsection{Análisis descriptivo de los tipos de algas}

% Como interesa conocer el efecto de la temporada del año en que fue tomada la muestra, el
% tamaño y la velocidad del río en el crecimiento de las algas, se realizó el mismo estudio
% estadístico con las muestras seccionadas según los parametros de interés. El resultado
% se muestra en los cuadros 2, 3 y 4

% Se puede observar que:

% \begin{itemize}
%   \item Respecto a la discriminación por temporada del año:
%   \begin{itemize}
%     \item ninguna de las medias de las variables se aleja mas de una desviación
%     estandar respecto al estudio anual realizado anteriormente
%     \item El máximo de n.a1. se mantiene considerablemente similar para otoño y primavera
%     disminuyendo un poco en invierno pero con un cambio significativo en verano
%   \end{itemize}
%   % \item Respecto a la discriminación por tamaño del río:
%   % \begin{itemize}
%   %   \item
%   % \end{itemize}
%   % \item Respecto a la discriminación por velocidad del río:
%   % \begin{itemize}
%   %   \item
%   % \end{itemize}
% \end{itemize}

% \begin{table}[!htbp]
% \centering
% \resizebox{\textwidth}{!}
%   {\begin{tabular}{ccccccccc}
%   \hline
%   Variable & Mínimo & Primer Cuartil & Mediana & Media   & Tercer Cuartil & Máximo & Desviación standar & Rango intercuartil \\
%   \hline
%   \textbf{Season: autumn } \\
%   n.a1. & -23.520 & 2.131 & 12.760 & 15.350 & 25.270 & 84.760 & 22.2877820820908 & 23.1357863793824 \\
%   n.a2. & -21.7600 & 0.3206 & 4.7340 & 4.4240 & 9.9490 & 22.2100 & 9.53807172423249 & 9.62808420774121 \\
%   n.a3. & -15.720 & 1.065 & 5.006 & 4.617 & 9.458 & 21.150 & 7.95198193798729 & 8.39291766239149 \\
%   n.a4. & -18.5800 & -0.3682 & 7.8890 & 6.5520 & 13.0800 & 34.0700 & 11.1376652255619 & 13.4462457682849 \\
%   \hline
%   \textbf{Season: spring } \\
%   n.a1. & -32.570 & -8.616 & 12.950 & 11.490 & 26.600 & 84.420 & 26.398381097064 & 35.2182039374182 \\
%   n.a2. & -8.359 & 1.236 & 9.796 & 9.647 & 15.650 & 34.530 & 10.2495393359626 & 14.4101463405427 \\
%   n.a3. & -14.5900 & -0.5069 & 6.5740 & 4.8130 & 8.7700 & 21.2400 & 7.50008737059129 & 9.27685873627981 \\
%   n.a4. & -11.340 & -1.231 & 2.748 & 5.892 & 7.881 & 39.550 & 11.0323424712034 & 9.11202331614136 \\
%   \hline
%   \textbf{Season: summer } \\
%   n.a1. & -32.050 & -1.034 & 9.964 & 12.310 & 24.420 & 61.910 & 19.0420386250236 & 25.4526490167935 \\
%   n.a2. & -23.220000 & 0.008411 & 8.079000 & 7.335000 & 15.190000 & 30.270000 & 11.673574625179 & 15.1771423974448 \\
%   n.a3. & -10.270 & 0.237 & 4.912 & 3.858 & 8.117 & 14.420 & 5.63427204434771 & 7.88043589405455 \\
%   n.a4. & -29.920 & -4.966 & 4.807 & 3.039 & 11.320 & 27.290 & 12.7991571863114 & 16.2908426954562 \\
%   \hline
%   \textbf{Season: winter } \\
%   n.a1. & -33.29000 & -0.07674 & 10.09000 & 11.88000 & 22.65000 & 81.64000 & 21.4693720953533 & 22.7263415748526 \\
%   n.a2. & -20.610 & -2.876 & 6.361 & 4.804 & 13.400 & 25.150 & 11.2600468747436 & 16.2708682610359 \\
%   n.a3. & -10.340 & -1.651 & 3.947 & 3.938 & 7.839 & 22.800 & 7.5118551736701 & 9.48951949735014 \\
%   n.a4. & -16.930 & -1.593 & 5.085 & 7.777 & 13.280 & 47.250 & 13.8552715343832 & 14.8704627681258 \\
%   \hline
%   \end{tabular}}
% \caption{Análisis descriptivo de los tipos de algas considerando la temporada del año}
% \end{table}

% \begin{table}[!htbp]
% \centering
% \resizebox{\textwidth}{!}
%   {\begin{tabular}{ccccccccc}
%   \hline
%   Variable & Mínimo & Primer Cuartil & Mediana & Media   & Tercer Cuartil & Máximo & Desviación standar & Rango intercuartil \\
%   \hline
%   \textbf{Size: large}\\
%   n.a1. & -23.520 & 2.108 & 14.730 & 13.220 & 24.770 & 61.910 & 17.5385766261161 & 22.6607988268709 \\
%   n.a2. & -3.911 & 3.856 & 9.988 & 10.140 & 16.330 & 27.510 & 7.824867996063 & 12.476446601254 \\
%   n.a3. & -5.076 & 2.408 & 4.959 & 5.060 & 8.581 & 13.610 & 4.45368479108391 & 6.17351690480806 \\
%   n.a4. & -18.580 & -2.008 & 3.317 & 4.243 & 10.480 & 37.730 & 11.5501586537458 & 12.4914388023028 \\
%   \hline
%   \textbf{Size: medium}\\
%   n.a1. & -33.290 & -3.781 & 7.021 & 6.445 & 16.810 & 53.020 & 18.1315601881389 & 20.5907375638207 \\
%   n.a2. & -23.220 & -1.436 & 7.265 & 6.194 & 14.910 & 30.270 & 11.3637948173828 & 16.3468066415998 \\
%   n.a3. & -15.7200 & 0.3243 & 5.5970 & 4.5860 & 8.9670 & 21.1500 & 7.3655652612440 & 8.6425764465757 \\
%   n.a4. & -29.920 & -1.731 & 9.353 & 8.857 & 17.890 & 47.250 & 15.4733197906767 & 19.6250851018339 \\
%   \hline
%   \textbf{Size: small}\\
%   n.a1. & -32.570 & 3.648 & 21.350 & 20.690 & 35.900 & 84.760 & 27.8400745200151 & 32.2527672102881 \\
%   n.a2. & -20.610 & -2.027 & 4.595 & 4.605 & 12.180 & 34.530 & 11.6672628124684 & 14.2096075475675 \\
%   n.a3. & -14.590 & -2.004 & 2.905 & 3.295 & 7.492 & 22.800 & 8.33551853161713 & 9.49586747420109 \\
%   n.a4. & -6.6260 & -0.2508 & 2.7840 & 3.0390 & 6.4890 & 18.3200 & 5.62394939653192 & 6.74012360680746 \\
%   \hline
%   \end{tabular}}
% \caption{análisis descriptivo de los tipos de algas considerando el tamaño}
% \end{table}

% \begin{table}[!htbp]
% \centering
% \resizebox{\textwidth}{!}
%   {\begin{tabular}{ccccccccc}
%   \hline
%   Variable & Mínimo & Primer Cuartil & Mediana & Media   & Tercer Cuartil & Máximo & Desviación standar & Rango intercuartil \\
%   \hline
%   \textbf{Speed: high}\\
%   n.a1. & -32.570 & -2.211 & 10.720 & 12.510 & 25.280 & 81.640 & 22.362496446141 & 27.496296033649 \\
%   n.a2. & -23.220 & -5.536 & 1.411 & 2.793 & 12.510 & 30.270 & 11.9146424743533 & 18.0432699230074 \\
%   n.a3. & -14.5900 & -0.5069 & 5.3420 & 5.1150 & 8.6300 & 22.8000 & 7.37467069781078 & 9.13675319783699 \\
%   n.a4. & -20.180 & -1.909 & 3.735 & 5.109 & 10.330 & 42.540 & 11.6422277057398 & 12.2361232177878 \\
%   \hline
%   \textbf{Speed: low}\\
%   n.a1. & -32.430 & -4.265 & 11.580 & 8.086 & 21.330 & 61.910 & 22.5755207242416 & 25.5986606246764 \\
%   n.a2. & -21.760 & 3.818 & 9.451 & 8.871 & 14.730 & 22.930 & 8.870517571768:  & 10.91441269776 \\
%   n.a3. & -15.720 & -2.170 & 3.947 & 2.782 & 7.696 & 20.490 & 7.64245481106906 & 9.86551333385625 \\
%   n.a4. & -18.580 & 0.592 & 7.508 & 9.610 & 18.030 & 47.250 & 14.8390475872179 & 17.4416540979921 \\
%   \hline
%   \textbf{Speed: medium}\\
%   n.a1. & -33.290 & 1.949 & 12.240 & 14.410 & 26.150 & 84.760 & 22.3205248475467 & 24.2021119890446 \\
%   n.a2. & -10.520 & 2.591 & 9.125 & 9.407 & 15.690 & 34.530 & 9.52337934629792 & 13.0966813131552 \\
%   n.a3. & -13.900 & 0.341 & 4.470 & 4.060 & 8.845 & 21.150 & 6.71300681328213 & 8.50423463954797 \\
%   n.a4. & -29.920 & -1.575 & 4.311 & 5.279 & 11.420 & 37.730 & 12.0226760782001 & 12.9905432529521 \\
%   \hline
%   \end{tabular}}
% \caption{análisis descriptivo de los tipos de algas considerando la velocidad del río}
% \end{table}

% \subsection{Gráficos}

%   \begin{figure}[H]
%     \centering
%     \begin{minipage}{.5\textwidth}
%       \centering
%       \includegraphics[scale=0.5]{../graficos/season_autumn}
%       \caption{tipos de algas para la temporada de otoño}
%     \end{minipage}
%     \begin{minipage}{.5\textwidth}
%       \centering
%        \includegraphics[scale=0.5]{../graficos/season_spring}
%       \caption{tipos de algas para la temporada de primavera}
%     \end{minipage}
%   \end{figure}

%   \begin{figure}[H]
%     \centering

%     \begin{minipage}{\textwidth}
%       \centering
%       \includegraphics[scale=0.5]{../graficos/season_summer}
%       \caption{tipos de algas para la temporada de verano}
%     \end{minipage}
%     \begin{minipage}{\textwidth}
%       \centering
%        \includegraphics[scale=0.5]{../graficos/season_winter}
%       \caption{tipos de algas para la temporada de invierno}
%     \end{minipage}
%   \end{figure}

%   \begin{figure}[H]
%     \centering

%     \begin{minipage}{\textwidth}
%       \centering
%       \includegraphics[scale=0.5]{../graficos/size_large}
%       \caption{tipos de algas para tamaño del rio grande}
%     \end{minipage}
%     \begin{minipage}{\textwidth}
%       \centering
%        \includegraphics[scale=0.5]{../graficos/size_medium}
%       \caption{tipos de algas para tamaño del rio mediano}
%     \end{minipage}
%   \end{figure}

%   \begin{figure}[H]
%     \centering

%     \begin{minipage}{\textwidth}
%       \centering
%       \includegraphics[scale=0.5]{../graficos/size_small}
%       \caption{tipos de algas para tamaño del rio pequeño}
%     \end{minipage}
%   \end{figure}

%   \begin{figure}[H]
%     \centering

%     \begin{minipage}{\textwidth}
%       \centering
%       \includegraphics[scale=0.5]{../graficos/speed_high}
%       \caption{tipos de algas para velocidad del rio alta}
%     \end{minipage}
%     \begin{minipage}{\textwidth}
%       \centering
%        \includegraphics[scale=0.5]{../graficos/speed_medium}
%       \caption{tipos de algas para velocidad del rio mediana}
%     \end{minipage}
%   \end{figure}

%   \begin{figure}[H]
%     \centering

%     \begin{minipage}{\textwidth}
%       \centering
%       \includegraphics[scale=0.5]{../graficos/speed_low}
%       \caption{tipos de algas para velocidad del rio baja}
%     \end{minipage}
%   \end{figure}

% \subsection{Intervalos de confianza para la media y la varianza de las algas más perjudiciales}
% \begin{table}[!htbp]
% \centering
% \resizebox{\textwidth}{!}
%   {\begin{tabular}{ccccccccc}
%   \hline
%   Variable & temporara & IC para media & IC para varianza \\
%   \hline
%   n.a2. & autumn & (0.09359729 , 8.75354831)\\
%   n.a2. & spring & (5.675403 , 13.618431)\\
%   n.a2. & summer & (2.531894 , 12.138108)\\
%   n.a2. & winter & (0.8273424 , 8.7811737)\\
%   n.a4. & autumn & (1.495763 , 11.608041)\\
%   n.a4. & spring & (1.617427 , 10.167100)\\
%   n.a4. & summer & (-2.227091  , 8.305367)\\
%   n.a4. & winter & (2.883516 , 12.670553)\\

%   \end{tabular}}
% \caption{Intervalos de confianza para la media y la varianza de las algas más perjudiciales para cada temporada del año}
% \end{table}

% % \subsection{Intervalos de confianza para la diferencia entre las medias de algas tipo2 (n.a2.) con las algas tipo4 (n.a4.)}
