\section{Descripcion del experimento}

Los datos presentados contienen indicadores de informacion de
Twitter tomados en 2013.
Cada fila de esta data representa siete dias de observacion de un
topico especifico
( por ejemplo, Coldplay ) considerando un par de dias luego de su
primera
observacion.
Se considera que un topico es un Buzz! si hay a lo menos 500
discusiones
activas por dia (en promedio, con respecto a la observacion inicial)
Todas las observaciones son independientes
En total hay 140.707 filas.
Cada fila contiene 11 atributos medidos a lo largo de siete dias:
\begin{itemize}
  \item{Columnas 1-7: NCD1 ... NCD7 numero de discusiones creadas
respecto al topico por dia}
  \item{Columnas 8-14: AI1 .. AI7 Incremento de nuevos autores
interactuando (Popularidad)}
  \item{Columnas 15-21: AL1 .. AL7 Nivel de atencion, es decir, cantidad
de autores por dia medido en escala intervalo (0,1)}
  \item{Columnas 22-28: BL1 .. BL7 Nivel de impacto (burst) medido como
el numero de discusiones creadas entre el la cantidad de discusiones  (NCD
/ NAD)}
  \item{Columnas 29-35: AC1 .. AC7 Numero de Contenedores Atomicos en las
redes sociales}
  \item{Columnas 36-42: AL1 .. AL7 Una medida de la atencion prestada en
las redes sociales}
  \item{Columnas 43-49: CS1 .. CS7 Dispersion de la contribucion al
topico}
  \item{Columnas 50-56: AT1 .. AT7 Interaccion de los autores, el
promedio de autores contribuyendo en el topico}
  \item{Columnas 57-63: NA1 .. NA7  Numero total de autores contribuyendo
en el topico por dia}
  \item{Columnas 64-70: DL1 .. DL7  Longitud de la discusion.}
  \item{Columnas 71-77: NAD1 .. NAD7 Cantidad de discusiones en el topico}
\end{itemize}
