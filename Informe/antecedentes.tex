\section{Antecedentes}

El analisis automatico de las redes sociales es uno de los topicos de
mayor movimiento en la estadistica y en las areas de Data Science en
la ultima decada. El volumen y velocidad con que se genera la
informacion hace extremadamente relevante poder identificar y
clasificar el comportamiento de las interacciones de los usuarios de
redes sociales.

Uno de los topicos en estudio es la deteccion de Topicos, o elementos
que pudieran ser de interes masivo en la red. Cada topico (por
ejemplo un hashtag en twitter) involucra una serie de interacciones
en diferentes discusiones (twits) entre diferentes autores que pueden
o no involucrar varios otros topicos o discusiones. Cada topico,
ademas, tiene un tiempo de vida durante el cual el topico muestra
mucha actividad.

Los datos del presente proyecto forman parte de una recoleccion de
datos usada para crear modelos que puedan predecir la aparicion de
Buzz's o topicos de alta relevancia en la red social. La base de
datos solo considera varios topicos que recibieron contribuciones por
varios dias, y se tomaron varias medidas sobre el topico durante
siete dias consecutivos. En general, un topico se considera un Buzz
si presenta a lo menos 500 contribuciones por dia.
